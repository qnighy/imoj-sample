\documentclass[10pt,a4j,notitlepage,uplatex]{jsarticle}
\title{Guess the number}

\begin{document}
\section{課題}

正の整数$N$が与えられる。関数\texttt{guess}を用いて、整数$K$の値を当てよ。

\section{関数}

ユーザーは関数\texttt{void game(int N)}を実装しなければならない。

\texttt{game}は関数\texttt{int guess(int value)}を高々$100$回呼び出すことができる。

\begin{itemize}
  \item \texttt{value}が$K$より小さいとき、\texttt{guess}は$-1$を返す。
  \item \texttt{value}が$K$より大きいとき、\texttt{guess}は$-1$を返す。
  \item \texttt{value}が$K$と等しいとき、\texttt{guess}は値を返さずにプログラムを終了する。
\end{itemize}

\section{制約}

全ての入力は次の制約を満たす。
\begin{itemize}
  \item $1 \leq N \leq 10^{8}$
  \item $0 \leq K < N$
\end{itemize}

\section{小課題}

\subsection{小課題1 (50点)}

この小課題に対する入力は次の制約を満たす。
\begin{itemize}
  \item $1 \leq N \leq 100$
\end{itemize}

\subsection{小課題2 (50点)}

追加の制約はない。

\end{document}
