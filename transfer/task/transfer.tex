\documentclass[10pt,a4j,notitlepage,uplatex]{jsarticle}
\title{Transfer the numbers}

\begin{document}
\section{課題}

プレイヤー$A$に非負整数$A, B$が与えられる。$A, B$をプレイヤー$B$に伝えるプログラムを作成せよ。

\section{関数}

\texttt{playerA}と\texttt{playerB}の二つのソースプログラムを実装せよ。これらのプログラムは、グローバル変数などを用いて直接やりとりを行ってはいけない。関数の外で変数を宣言する場合や、実装しなければいけない関数以外の関数を定義する場合は、\texttt{static}を付けること。

\texttt{playerA}には関数\texttt{void playerA(int A, int B)}を実装しなければならない。

\texttt{playerA}は関数\texttt{void send(int b)}を高々$40$回呼び出すことができる。$b$は$0$か$1$のいずれかでなければならない。

\texttt{playerB}には関数\texttt{void playerB(int N, int data[])}を実装しなければならない。$N$は\texttt{send}が呼び出された回数である。配列\texttt{data}には\texttt{send}に渡された$b$の値が入っている。

\texttt{playerB}は関数\texttt{void answer(int X, int Y)}を呼び出さなければならない。\texttt{answer}が呼ばれると、プログラムは終了する。$X = A, \ Y = B$のとき、正解となる。


\section{制約}

全ての入力は次の制約を満たす。
\begin{itemize}
  \item $0 \leq A$
  \item $0 \leq B$
  \item $A + B < 2 \times 10^{6}$
\end{itemize}

\section{小課題}

\subsection{小課題1 (50点)}

この小課題に対する入力は次の制約を満たす。
\begin{itemize}
  \item $0 \leq A$
  \item $0 \leq B$
  \item $A + B < 10^{6}$
\end{itemize}

\subsection{小課題2 (50点)}

追加の制約はない。

\end{document}
